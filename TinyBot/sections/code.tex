\documentclass[../TinyBot.tex]{subfiles}
\begin{document}

\section{Code}

Thinking back to the H-Bridge, to control the direction of the motor we want to control which
switches are closed and which are open. Switches are generally active low - which means that 
they are off by default. To turn them on, or ``close them'', we want to set pins to high. \\

To make it easier on ourselves, let's name these switchs as constants. Let's put these
at the very top of our script so that every function we write later can access these constants. \\

\begin{lstlisting}
//Motor 1
#define MOTOR_PIN_1 6   // Pin 15 of L293
#define MOTOR_PIN_2 5   // Pin 10 of L293
//Motor 2
#define MOTOR_PIN_3 9   // Pin  7 of L293
#define MOTOR_PIN_4 8   // Pin  2 of L293
\end{lstlisting}

The keyword \lstinline[]!#define! means that the value cannot be changed later in the code.
Annoyingly, when we define constants like this, we have to omit the equals symbol and
semi-colon.\footnote{Research the \lstinline[]!C++! preprocessor in your spare time to
finc out why} The value of the constants can be changed to any digital pin number on the
Arduino, though make sure that the value matches the physical pin used. \\

Next we need to set the pin mode of the pins we're using to interact with the H-Bridge.

\begin{lstlisting}
void setup(){
    //Set pins as outputs
    pinMode(MOTOR_PIN_1, OUTPUT);
    pinMode(MOTOR_PIN_2, OUTPUT);
    pinMode(MOTOR_PIN_3, OUTPUT);
    pinMode(MOTOR_PIN_4, OUTPUT);
}
\end{lstlisting}

To drive forward, we want the motors to turn in the same direction and so we want to
enable \lstinline[]!IN1! and \lstinline[]!IN3!. \\

\begin{lstlisting}
void setup() {
    // set pinMode as done above

    digitalWrite(MOTOR_PIN_1, HIGH);
    digitalWrite(MOTOR_PIN_2, LOW);
    digitalWrite(MOTOR_PIN_3, HIGH);
    digitalWrite(MOTOR_PIN_4, LOW);
    delay(2000); // wait for 2 seconds

    // Turn motors off
    digitalWrite(MOTOR_PIN_1, LOW);
    digitalWrite(MOTOR_PIN_3, LOW);
}
\end{lstlisting}
As this code is in \lstinline[]!setup()!, it will run once, i.e. the robot will drive forwards
for 2 seconds, then stop. If you want the robot to drive forward continuously, then move the
digital writes to \lstinline[]!loop()!. Make sure to put the robot on the floor so that it
doesn't drive off the desk. \\

As you can imagine, having to set all 4 motor pins individually when controlling the robot
can get quite tedious. A simple solution to this is to create functions for driving in
each direction. We're going to assume that the left motor is motor 1, and the right motor is
motor 2. \\

\begin{lstlisting}
void driveForward() {
    // motor 1 - left
    digitalWrite(MOTOR_PIN_1, HIGH);
    digitalWrite(MOTOR_PIN_2, LOW);
    // motor 2 - right
    digitalWrite(MOTOR_PIN_3, HIGH);
    digitalWrite(MOTOR_PIN_4, LOW);
}

void driveBackwards() {
    // motor 1 - left
    digitalWrite(MOTOR_PIN_1, LOW);
    digitalWrite(MOTOR_PIN_2, HIGH);
    // motor 2 - right
    digitalWrite(MOTOR_PIN_3, LOW);
    digitalWrite(MOTOR_PIN_4, HIGH);
}

void turnLeft() {
    // turn off right motor, and drive left motor forwards
    // motor 1 - left
    digitalWrite(MOTOR_PIN_1, HIGH);
    digitalWrite(MOTOR_PIN_2, LOW);
    // motor 2 - right
    digitalWrite(MOTOR_PIN_3, LOW);
    digitalWrite(MOTOR_PIN_4, LOW);
}

void turnRight() {
    // turn of left motor, and drive right motor forwards
    // motor 1 - left
    digitalWrite(MOTOR_PIN_1, LOW);
    digitalWrite(MOTOR_PIN_2, LOW);
    // motor 2 - right
    digitalWrite(MOTOR_PIN_3, HIGH);
    digitalWrite(MOTOR_PIN_4, LOW);
}

void stop() {
    // motor 1 - left
    digitalWrite(MOTOR_PIN_1, LOW);
    digitalWrite(MOTOR_PIN_2, LOW);
    // motor 2 - right
    digitalWrite(MOTOR_PIN_3, LOW);
    digitalWrite(MOTOR_PIN_4, LOW);
}
\end{lstlisting}

\end{document}